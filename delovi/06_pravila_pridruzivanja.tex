\section{Pravila pridruzivanja}
\label{sec:PravilaPridruzivanja}

// TODO uvodni tekst


\subsection{Zavisnosti izmedju \v{z}anrova}
\label{subsec:AprioriZavisnostZanrova}

Na slici \ref{fig:aprioriZanrovi} prikazano je 10 najizra\v{z}ajnijih pravila sortirana prvo po podr\v{s}ci, a zatim po lift meri, opadaju\'c{}e.

\begin{figure}[H]
    \footnotesize
    \centering
    \begin{tabular}{|c|c|c|c|c|}
        \hline
        Podr\v{s}ka & Pouzdanost & Lift & Glava pravila  & Telo pravila \\
        \hline
        0.2646 & 0.7404 & 1.5567 & rock & pop \\
        0.2646 & 0.5563 & 1.5567 & pop & rock \\
        0.2248 & 0.9930 & 2.0878 & rock & classic \\
        0.2163 & 0.8176 & 3.6114 & classic & pop, rock \\
        0.2163 & 0.9623 & 2.6926 & pop & rock, classic \\
        0.2163 & 1.0000 & 2.1026 & rock & pop, classic \\
        0.1612 & 0.9967 & 2.0957 & rock & indie \\
        0.1405 & 0.9170 & 5.9225 & british & uk \\
        0.1405 & 0.9075 & 5.9225 & uk & british \\
        0.1373 & 0.5582 & 1.1736 & rock & american, classic \\
        \hline
    \end{tabular}
    \caption{Rezultati Apriori algoritma koji pokazuju zavisnost medju \v{z}anrovima}
    \label{fig:aprioriZanrovi}
\end{figure}


\subsection{Zavisnost \v{z}anra od decenije}
\label{subsec:AprioriZavisnostZanraOdDecenije}

Na tabeli \ref{fig:aprioriDecade} prikazana su dobijena pravila sortirana prvo po podr\v{s}ci, a zatim po lift meri, opadaju\'c{}e.
Kako se kasnije vidi na slici \ref{fig:YearDuration}, najve\'c{}i deo skupa obradjenih podataka pripada dvehiljaditim godinama. Zbog ovoga se dobijeni rezulati koncentri\v{s}u na ovoj deceniji. Kori\v{s}enje ve\'c{}eg i raznovrnijeg skupa podataka bi doneo druga\v{c}ije rezultate.

\begin{figure}[H]
    \footnotesize
    \centering
    \begin{tabular}{|c|c|c|c|c|}
        \hline
        Podr\v{s}ka & Pouzdanost & Lift & Glava pravila  & Telo pravila \\
        \hline
        0.0716 & 0.8133 & 1.6369 & 00s & pop, chart \\
        0.0525 & 0.7857 & 1.5815 & 00s & rnb, hop, hall, dance, hip \\
        0.0764 & 0.7385 & 1.4864 & 00s & dance \\
        0.0615 & 0.6824 & 1.3734 & 00s & rnb \\
        0.0615 & 0.6667 & 1.3419 & 00s & hop hip \\
        0.0541 & 0.6000 & 1.2077 & 00s & metal \\
        0.0530 & 0.5917 & 1.1910 & 00s & rock, alternative \\
        0.0578 & 0.5619 & 1.1309 & 00s & alternative \\
        0.0896 & 0.5541 & 1.1153 & 00s & indie \\
        0.0891 & 0.5526 & 1.1123 & 00s & rock, indie \\
        \hline
    \end{tabular}
    \caption{Rezultati Apriori algoritma koji pokazuju zavisnost izmedju \v{z}anrova i decenija tokom koje su bili popularni}
    \label{fig:aprioriDecade}
\end{figure}


\subsection{Zavisnost \v{z}anra od lokacije}
\label{subsec:AprioriZavisnostZanraOdLokacije}

Lokacija autora je atribut \v{c}ije ja vrednost \v{c}esto nepostoju\'c{}a. A ukoliko jeste, iste lokacije su na razli\v{c}itim pesmama druga\v{c}ije napisane. Na primer, jedna pesma ima lokaciju \emph{London}, a druga \emph{London, UK}. Ovakvi podaci su doveli do lo\v{s}ih rezultata istra\v{z}ivanja.
