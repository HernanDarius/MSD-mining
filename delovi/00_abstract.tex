
\abstract{
U ovom radu je opisana nova tehnika neograničene provere modela softvera za pronalazak grešaka programa pisanih u programskom jeziku C, koristeći inkrementalne SAT rešavače. Od polaznog programa se pravi \emph{DimSpec} formula. \emph{DimSpec} formula predstavlja konjunkciju četiri KNF formule koje kodiraju početno, krajnje i svako tranziciono stanje programa, kao i prelaze između susednih stanja programa. Proces dobijanja \emph{DimSpec} formule od izvornog C koda se odvija u više koraka. Prvo se C kod prevodi u \emph{LLVM} module, koji predstavljaju međureprezentaciju između C koda i mašinskog koda. Ti moduli se zatim prevode u \emph{DimSpec} formulu korišćenjem alata \emph{LLUMC} tako da ciljno stanje predstavlja stanje greške. Tako dobijena \emph{DimSpec} formula se može rešiti pomoću inkrementalnih SAT rešavača ili \emph{IC3} algoritmom za proveru invarijanti. Rešenje prestavlja dokaz neispravnosti programa jer je pronađen način da se iz polaznog stanja stigne u stanje greške. Ovakvo kodiranje proširuje funkcionalnost tradicionalne \emph{ograničene provera modela sofrvera} jer pokriva velike i beskonačne petlje, uz održavanje pristojnih vremenskih performansi.
}
